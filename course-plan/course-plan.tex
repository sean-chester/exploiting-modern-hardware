\documentclass[12pt,letterpaper]{article}

\usepackage[letterpaper, margin=0.75in]{geometry}
\usepackage{newtxtext,newtxmath} % times new roman font
\usepackage{url}
\usepackage[dvipsnames]{xcolor}

% shrink space between reference list entries as per:
% https://tex.stackexchange.com/a/296274/84485
\usepackage{etoolbox}
\patchcmd{\thebibliography}
  {\settowidth}
  {\setlength{\itemsep}{0pt plus 0.1pt}\settowidth}
  {}{}
\apptocmd{\thebibliography}
  {}
  {}{}

\RequirePackage{filecontents}
\begin{filecontents}{\jobname.bib}
@misc{aresdb,
  author = {{\relax Uber Engineering}},
  title = {Introducing {AresDB}: Uber's {GPU}-Powered Open Source, Real-time Analytics Engine},
  howpublished = {\url{https://eng.uber.com/aresdb}},
  year = {2019},
  note = {Accessed: 2019-08-14}
}

@InProceedings{opioid,
  title = {Leveraging {Twitter} and {Neo4j} to Study the Public Use of Opioids in the {USA}},
  author = {Soni, Disha and Ghanem, Thanaa and Gomaa, Basma and Schommer, Jon},
  booktitle = {Proc. GRADES-NDA},
  year = {2019},
  pages = {1--5},
  note = {\url{https://doi.org/10.1145/3327964.3328501}}
}

@InProceedings{linkedin,
  author = {Andrew Carter and others},
  title = {Nanosecond Indexing of Graph Data With Hash Maps and VLists},
  booktitle = {Proc. ACM SIGMOD},
  year = {2019},
  pages = {623--625},
}

@InProceedings{word2vec,
  author = {Tomas Mikolov and others},
  title = {Distributed Representations of Words and Phrases and their Compositionality},
  booktitle = {Proc. NIPS},
  year = {2013},
  pages = {},
}

@InProceedings{hetero,
  author = {Raul Castro Fernandez and others},
  title = {Termite: a system for tunneling through heterogeneous data},
  booktitle = {Proc.  aiDM@SIGMOD},
  year = {2019},
  pages = {},
}
\end{filecontents}

\newcommand{\csection}[1]{\section*{\centering #1}\vspace{-0.25em}}
\title{Course Plan: CSC 586C\\Data Management on Modern Computer Architectures}
\author{\vspace{-7em}Instructor: {\em Sean Chester}~~}
\date{Term: Spring 2020}

\begin{document}
\maketitle


\csection{Overview}
\hspace{0.06\textwidth}
\parbox{0.88\textwidth}{%
An in-depth study of architecture-conscious algorithm design and analysis that challenges students to connect theory and systems. Students are expected to design and refine algorithms for several shared-memory parallel platforms, including super-scalar, hyper-threaded cores, multi-core and multi-socket (NUMA) systems, and general purpose graphics processing units (GPUs).
}

\vspace{1em}
\hspace{0.15\textwidth}\parbox{0.7\textwidth}{\textcolor{gray!25}{\hrule}}
\vspace{-1em}

\csection{Pedagogical Objectives}
{\small \em Note: Enrolment consists of 17 CS undergrad students, and 12 CS grad students}

\medskip
By the end of the course, the students should:
\begin{enumerate}
  \item Be ready to conduct research in shared-memory parallel algorithms
  \item Be able to compare and contrast computational models for different computer architectures
  \item Have hands-on programming experience with microbenchmarking and empirical analysis of algorithms on several shared-memory parallel platforms
  \item Know how to iteratively diagnose and optimise slow implementations and use this to inform algorithm design
  \item Have produced a portfolio-worthy project that exploits and effectively analyses shared-memory parallelism
  \item Have expanded their programming skills in modern C++ and/or similar systems-oriented programming languages (e.g., CUDA, Rust)
\end{enumerate}


\vspace{-0.5em}
\hspace{0.15\textwidth}\parbox{0.7\textwidth}{\textcolor{gray!25}{\hrule}}
\vspace{-1em}

\csection{Course Content}
The course will be split into five modules, each corresponding to a different architecture, and each gradually introducing a higher degree of parallelism.

\subsubsection*{Introduction and Motivation (1 week)}
\begin{itemize}
  \item Reviewing expectations for the course
  \item Motivation for studying architecture-conscious algorithm design, e.g., Sutter articles
  \item Basic introduction of, e.g., microbenchmarking, c++ functional-style programming, asymptotic analysis, concept of latency hiding and its relationship to cpu vs mem cost
\end{itemize}

\subsubsection*{Super-scalar cores (3 weeks)}
\begin{itemize}
  \item Super-scalar computation model
  \item Parallelism on a single core: ILP/super-scalar, hyper-threading, relationship to branch prediction, locality
  \item SoA vs AoS, death to Java :)
  \item Tiling and other techniques to improve (spatial/temporal) locality
  \item Rigorous analysis of single-core performance (e.g., hardware counters) and hot spot detection
\end{itemize}

\subsubsection*{Multi-core and multi-socket (3 weeks)}
\begin{itemize}
  \item Multicore computation model, PRAM asymptotic model and its limitations, Amdahl's Law
  \item Architectural diagrams for multi-core, multi-socket layouts, memory transfer costs
  \item Frameworks for multi-core, e.g., c++ threads, openmp
  \item Analysing and diagnosing parallel performance, false sharing, cache invalidation, parallel scalability, work efficiency
  \item Algorithmic considerations: shared data, constness
\end{itemize}

\subsubsection*{GPUs (3 weeks)}
\begin{itemize}
  \item GPU computation model, thread blocks, warps, cache hierarchy, latency hiding, saturation
  \item CUDA, OpenCL, Nvidia simulator + profiler, thrust
  \item PCIe and hybrid cpu-gpu computation
  \item Example algorithms
\end{itemize}

\subsubsection*{Advanced Topics (2 weeks)}
Likely to include one-to-two days each on (algorithms for) :
\begin{itemize}
  \item Quantum computing
  \item FPGAs
  \item PIM/DPU's
  \item Configurable spatial accelerator
  \item Extension to cloud computing or distributed systems?
\end{itemize}


\vspace{1em}
\hspace{0.15\textwidth}\parbox{0.7\textwidth}{\textcolor{gray!25}{\hrule}}
\vspace{-1em}

\csection{Student Evaluation}
\subsubsection*{Team projects}
The course will be largely based on one research-based term project done in small teams (2 grads or 3 undergrads), but with evaluation doled out on interim progress reports after each module. A presentation for grad students will differentiate their work. The intent is that these should be (at least) nearly publishable on some novel problem that each group tackles.

\medskip\noindent
Each interim report (after intro, super-scalar, multicore, gpu) should apply concepts from the latest module, but the final report can specialise. Every group should have a unique problem to study, but these can be identified later into the course.

\medskip\noindent
Pedagogically, the intent is also that students have to keep revisiting the same evolving code base over the course of the semester as they expand features.

\subsubsection*{Midterm exam}
A midterm exam, shortly after the end of the GPU module, will serve to separate group marks and ensure that each student individually has mastered the concepts. The exam will be comprehensive over the first four modules but require synthesis of the course's core concepts. This primarily assesses pedagogical objectives 2 and 6.


%\nocite{hetero}
%\pagenumbering{arabic}
\bibliographystyle{acm}
\bibliography{course-plan}


\end{document}
