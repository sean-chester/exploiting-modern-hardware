% !TeX encoding = UTF-8
% !TeX program = LuaLaTeX
% !TeX spellcheck = en_CA

\usepackage{listings}

\usepackage{xcolor}
\definecolor{ntnublue}{HTML}{124990}
\usepackage{tcolorbox}

\mode<presentation>
{
  \usetheme{metropolis}
  % \usecolortheme{whale}

  \setbeamercolor*{palette primary}{use=structure,fg=white,bg=ntnublue}
  \setbeamercolor*{palette secondary}{use=structure,fg=white,bg=ntnublue}
  \setbeamercolor*{palette tertiary}{use=structure,fg=white,bg=ntnublue}
  \setbeamercolor*{palette quartenary}{use=structure,fg=white,bg=ntnublue}
  \setbeamercolor*{frametitle}{use=structure,fg=black,bg=white}

  \setbeamertemplate{itemize item}[triangle]
  \setbeamercolor{itemize item}{fg=ntnublue}
  \setbeamercolor{block title}{fg=ntnublue}
  \setbeamercolor{block body}{fg=ntnublue!75}
}



% If you have a file called "university-logo-filename.xxx", where xxx
% is a graphic format that can be processed by latex or pdflatex,
% resp., then you can add a logo as follows:
 \pgfdeclareimage[height=0.6cm]{ntnu-logo}{../latex/fig/ntnu_logo.png}
 \pgfdeclareimage[height=1.1cm]{ntnu-logo-large}{../latex/fig/ntnu_logo.png}

% png figures used in slides
 \pgfdeclareimage[height=4cm]{cpu-vs-gpu}{../latex/fig/gpu-vs-cpu.png}
 \pgfdeclareimage[height=5cm]{lectures}{../latex/fig/lecture-schedule.png}
 \pgfdeclareimage[height=2cm]{aresdb-logo}{../latex/fig/aresdb-logo.png}
 \pgfdeclareimage[height=5cm]{un-sdg}{../latex/fig/un-sdg84.png}
 \pgfdeclareimage[height=3.5cm]{twitter}{../latex/fig/twitter.png}
 \pgfdeclareimage[height=3.5cm]{mem-speed}{../latex/fig/mem-speed.png}
 \pgfdeclareimage[height=2.5cm]{btree}{../latex/fig/Bplustree.png}
 \pgfdeclareimage[height=6cm]{volta-specs}{../latex/fig/volta-specs-core.png}
 \pgfdeclareimage[height=4cm]{volta-mem}{../latex/fig/volta-specs-mem.png}
 \pgfdeclareimage[height=3cm]{stalls}{../latex/fig/stalls.png}
 \pgfdeclareimage[height=2.5cm]{cpi}{../latex/fig/cpi-distr.png}
 \pgfdeclareimage[height=2cm]{flights}{../latex/fig/flights.png}
 \pgfdeclareimage[height=2cm]{webgraph}{../latex/fig/graph-compression.png}

% courtesy of http://bloerg.net/2012/06/21/customizing-the-frametitle-of-beamer-presentation.html
% \setbeamertemplate{frametitle}
% {
%     \nointerlineskip
%     \begin{beamercolorbox}[sep=0.3cm,ht=1.8em,wd=\paperwidth]{frametitle}
%         \vbox{}\vskip-2ex%
%         \strut\insertframetitle\strut
%         \hfill
%         \raisebox{-.075cm}
%         {\pgfuseimage{ntnu-logo}}
%         \vskip-1.5ex%
%     \end{beamercolorbox}
% }



\usepackage[utf8x]{inputenc}
%\usepackage[T1]{fontenc} <--- breaks \bf \em, etc.
\usepackage{fontspec}

 % for emoji
% \newfontfamily\DejaSans{DejaVu Sans}

\usepackage{amsmath,amsthm}


\usepackage{rotating}
\usepackage{tikz,pgfplots,pgfplotstable}
\usetikzlibrary{shapes,arrows,fit,calc,positioning,patterns,decorations.markings,decorations.pathreplacing}

\tikzset{
  double arrow/.style args={#1 colored by #2 and #3}{
    -stealth,line width=#1,#2, % first arrow
    postaction={draw,-stealth,#3,line width=(#1)/3,
                shorten <=(#1)/3,shorten >=2*(#1)/3}, % second arrow
  }
}

\tikzstyle{innerWhite} = [semithick, white,line width=1.4pt, shorten >= 4.5pt]
\tikzstyle{vecArrow} = [thick, decoration={markings,mark=at position
   1 with {\arrow[semithick]{open triangle 60}}},
   double distance=1.4pt, shorten >= 5.5pt,
   preaction = {decorate},
   postaction = {draw,line width=1.4pt, white,shorten >= 4.5pt}]

\usepackage[normalem]{ulem}
\usepackage{enumerate}

\usepackage{subcaption}

  
\author[Sean Chester]{Sean Chester\\\small\url{schester@uvic.ca}}
\date{}
\institute[UVic, 7 Jan 2020]{
CSC 485C/586C: Data Management on Modern Computer Architectures\\
ECS 104, University of Victoria\\
\when% define in main file before including this
}

\subject{UVic CSC 485C/586C course, presented in 50 minutes}

\keywords{csc 485C, csc 586C, outline, plan, expectations}

% \pdfinfo{
%   /Author (Sean Chester)
% }
